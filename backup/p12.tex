This paper explores the use of features that capture speakers' past turn-taking behavior in predicting whether there will be a turn transition.  These summary features include (a) relative turn length: how the current turn under construction compares to the current speaker's average turn length; and (b) relative floor control: the percentage of time that the current speaker has held the floor.  We include two versions of each, one based on time, and one based on number of words.
%
Relative turn length should capture whether one or both of the speakers tends to hold the turn over multiple utterances, while relative floor control captures whether one speaker is dominating the conversation.  Both of these factors should influence who will speak next.

In evaluating our model on data from the Switchboard corpus, we find that our summary features on their own do better than just using the previous speech act (accuracy of $66.14\%$ vs $60.26\%$).  We also find that when we add these features to a model that uses the last two speech acts, we also see an improvement ($76.05\%$ vs $74.43\%$).  These results show the potential of modeling speakers' past turn-taking behavior in predicting upcoming turn-transitions.  Better modeling of turn-taking should lead to more natural and efficient spoken dialogue systems.
