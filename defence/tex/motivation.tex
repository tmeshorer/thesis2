`% !TeX root = ../defense.tex

\section{Motivation}
\frame{\sectionpage}




\begin{frame}{Current Issues}
    \begin{enumerate}[<+->]\itemsep9pt
      \item For a natural conversation between human and machine, we want to conform
            to human to human turn taking system (Sacks et al, 1978)
      \item In Human-Human conversations conversant predict (Sacks et al, 1978) or
            signal (Duncan 1972) each other on coming turn transition
      \item {
        Timeouts leads to poor user interaction(Arsikere et al, 2015)
        \begin{itemize}
            \item Not effective in noisy environment
            \item too little - machine barge in during intra turn pause.
            \item too much - user waiting for the machine.
        \end{itemize}
      }
      \item {
        Turn transition prediction based on local features improve turn taking but still
        do not match human performance.
        \begin{itemize}
            \item Syntactic (Sacks et al 1978,De Ruiter et al. 2006)
            \item Prosodic (Ford 1996,Stolcke 2002,Ferrer 2003)
            \item Pragmatic (Ford 2001)
        \end{itemize}
      }
    \end{enumerate}
\end{frame}

\begin{frame} {Thesis Statement}
 \begin{center}

        \Large{Conversant's past behavior can help predict turn transitions}\\
        \vspace{10mm}
        \Large{Past behavior represented by Summary features}
 \end{center}
 \end{frame} 
 

\begin{frame} {Acknowledgement}
 \begin{center}
       \begin{enumerate}[<+->]\itemsep9pt
           \item This work was partially funded by the National Science Foundation under grant IIS-1321146.
           \item This thesis is based on a paper that was submitted and presented at Interspeech 2016.
           \item Thesis advisor and collaborator : Prof. Peter Heeman. 
       \end{enumerate}
 \end{center}
\end{frame}
