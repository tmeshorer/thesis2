`% !TeX root = ../defense.tex

\section{Motivation}
\frame{\sectionpage}




\begin{frame}{Current Issues}
    \begin{enumerate}[<+->]\itemsep9pt
      \item For a natural conversation between human and machine, we want to conform
            to human to human turn taking system (Sacks et al, 1978)
      \item In Human-Human conversations conversant predict (Sacks et al, 1978) or
            signal (Duncan 1972) each other on coming turn transition
      \item Human-Machine conversations mainly use time-out for predicting turn transition.      
      \item {
        However, timeouts leads to poor user interaction(Arsikere et al, 2015)
        \begin{itemize}
            \item Not effective in noisy environment
            \item too little - machine barge in during intra turn pause.
            \item too much - user waiting for the machine.
        \end{itemize}
      }
      \item {
        Recent research tries to use utterance local features but still do not match human performance.
        \begin{itemize}
            \item Syntactic (Sacks et al 1978,De Ruiter et al. 2006)
            \item Prosodic (Ford 1996,Stolcke 2002,Ferrer 2003)
            \item Pragmatic (Ford 2001)
        \end{itemize}
      }
    \end{enumerate}
\end{frame}

\begin{frame} {Thesis Statement}
   \begin{enumerate}[<+->]\itemsep9pt
         \item Conversant's past behavior can help predict turn transitions
         \item Past behavior represented by Summary features
         \begin{itemize}
            \item Relative turn length: current turn length so far (in seconds and words) relative to the speaker's average turn length
            \item Relative Floor Control: the speaker's control of the conversation floor (in seconds and words) relative to the total conversation length
            \item Computed for every dialog act.
         \end{itemize}
     \end{enumerate}
 \end{frame} 
 