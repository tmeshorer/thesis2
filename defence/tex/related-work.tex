`% !TeX root = ../defense.tex

\section{Related Work}
\frame{\sectionpage}


\begin{frame}{Turn taking in human-human conversations}
    \begin{enumerate}
       \item (Duncan 1972) Some signals and rules for taking speaking turns in conversations
          \begin{itemize}
            \item Intonation : terminate the final clause with rising or falling pitch.
            \item Drawl on the final syllable
            \item Body motion,
            \item Sociocentric sequence ("you know")
            \item Drop in pitch or loudness
            \item Syntax - completion of grammatical clause , involving subject-predicate combination.
          \end{itemize}
       \item (Sacks et al, 1974) A Simplest Systematics for the Organization of Turn-Taking for Conversation
          \begin{itemize}
            \item Introduced Turn Construction Unit (TCU) (word, phrase, clause, sentence)
            \item Defined Transition Relevance Place (TRP) between TCUs. Defined a local turn allocation system which is operational in each TRP
            \begin{itemize}
                \item Current speaker selects the next conversant;
                \item If the current speaker did not select, any of the listeners can self select; or
                \item If neither of the previous two cases apply, the current speaker continue.
            \end{itemize}
          \end{itemize}
       \item Later research focused on local features
        \begin{itemize}
            \item Syntactic (Sacks et al 1978,De Ruiter et al. 2006)
            \item Prosodic (Ford 1996,Stolcke 2002,Ferrer 2003)
            \item Pragmatic (Ford 2001)
        \end{itemize}
    \end{enumerate}
\end{frame}

\begin{frame} {Turn taking in Spoken dialogue systems I }
   \begin{enumerate}
         \item Early systems used time out based on speech signal threshold. Fixed length of 500ms (Arsikere et al, 2015)
         \item A. Raux and M. Eskenazi. (2009) - A Finite-State Turn-Taking Model for Spoken Dialog System
             \begin{itemize}
                \item Used a 6 state non deterministic state machine to model the turn taking system between the user and the system.
                \item Define a set of possible actions that both user and the system can take:
                      Grab the floor, Release the floor, Wait while not claiming the floor, and Keep
                      the floor.   
                \item Created a cost matrix to assign cost to each action in the state machine.
                \item Applied probabilistic decision theory principle of selecting the action with lowest expected cost
                \item Improvement of 29.5\% over the fixed timeout.    
             \end{itemize}  
         \item Selfridge and Heeman (2010) - A Bidding Approach to Turn-Taking
         \begin{itemize}
            \item Utterance importance is a driving force behind turn-taking
            \item Conversant measures the importance of their potential contribution when negotiating
                  the right to the conversation floor.
            \item Use Reinforcement Learning to map a given situation to the optimal utterance and bidding behavior  
             \item In relation to our work, conversants might use summary feature to signal their bid.        
         \end{itemize}
                 
   \end{enumerate}
 \end{frame}

\begin{frame} {Turn taking in Spoken dialogue systems II }
   \begin{enumerate}
      \item Gravano and Hirschberg (2011) - Turn-taking cues in task-oriented dialogue
         \begin{itemize}
            \item Impressive study of local utterance features as cues (signals) for turn taking.
            \item Formalized and verified Ducan work on signaling.
            \item Defined an IPU - Inter Pausal Unit. Max Sequence of words surrounded by silence.
            \item Tested over 200 features.  prosodic , syntactic , IPU duration, speaking rate.
            \item Found the combining features can improve turn transition prediction.
            \item Our work can complement the local features.
         \end{itemize}
        
         \item N. Guntakandla and R. Nielsen (2015) - Modelling turn-taking in human conversations
         \begin{itemize}
            \item Used dialog act as local feature.
            \item Tested prediction based on Bigram, Trigram and 4-gram of dialog acts.
            \item We based our base models on current and previous dialog acts.
            \item We used the same corpus, but mapped the dialog act types from 148 to 9.
         \end{itemize}
   \end{enumerate}
 \end{frame}

