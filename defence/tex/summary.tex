`% !TeX root = ../defense.tex

\section{Summary}
\frame{\sectionpage}

\begin{frame}{Conclusion}
   \begin{enumerate}[<+->]\itemsep9pt
      \item Summary features do provide improvement over local features. 
      \item However, the affect for our data is the opposite of our initial assumption
              \begin{itemize}
                \item Short turn (Low RTL) leads to turn change
                \item In long turn the speaker will actually hold the floor.
               \end{itemize}
    \end{enumerate}
\end{frame}{}

\begin{frame}{Future work}
   \begin{enumerate}[<+->]\itemsep9pt
      \item Combine the summary features with other local features (semantic/prosadic)
      \item Test the hypothesis on another type of corpus (for example task based corpus)
      \item Instead of measuring the affect from the start of the conversation, use moving averages with different window length.
      \item Perform the experiments where back channels are not considered as turn change. 
      \item In general, any local features can be turned into a summary feature by taking the avarage
            over past turn. Hence this area of research can be expanded to other local features.
      
          
    \end{enumerate}
\end{frame}{}



\begin{frame} {Acknowledgement}
 \begin{center}
       \begin{enumerate}[<+->]\itemsep9pt
           \item This work was partially funded by the National Science Foundation under grant IIS-1321146.
           \item This thesis is based on a paper that was submitted and presented at Interspeech 2016.
           \item Thesis advisor and collaborator : Prof. Peter Heeman.
       \end{enumerate}
 \end{center}
\end{frame}

