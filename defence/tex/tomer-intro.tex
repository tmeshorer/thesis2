`% !TeX root = ../defense.tex

\section{Motivation}
\frame{\sectionpage}

\begin{frame}{Current Issues}
    \begin{enumerate}[<+->]\itemsep9pt
      \item For a natural conversation between human and machine, we want to conform
            to human to human turn taking system (Sacks et al, 1978)
      \item In Human-Human conversations conversant predict (Sacks et al, 1978) or
            signal (Duncan 1972) each other on coming turn transition
      \item {
        Timeouts leads to poor user interaction(Arsikere et al, 2015)
        \begin{itemize}
            \item Not effective in noisy environment
            \item too little - machine barge in during intra turn pause.
            \item too much - user waiting for the machine.
        \end{itemize}
      }
      \item {
        Turn transition prediction based on local features improve turn taking but still
        do not match human performance.
        \begin{itemize}
            \item Syntactic (Sacks et al 1978,De Ruiter et al. 2006)
            \item Prosodic (Ford 1996,Stolcke 2002,Ferrer 2003)
            \item Pragmatic (Ford 2001)
        \end{itemize}
      }
    \end{enumerate}
\end{frame}

\begin{frame} {Goal of Work}
        Use summary features which are computed over the speaker history in
        the conversation to better predict turn transition.
\end{frame}

\begin{frame}{Theoretical Model}

%\begin{figure}
% \centering
% \includegraphics[scale=0.3]{c2.png}
% \end{figure}

  \begin{alertblock}{Conversation}
    { $\ldots  s_{i-2}, d_{i-2}, s_{i-1} , d_{i-1}, s_i, d_i  \ldots$  }
  \end{alertblock}
  \begin{alertblock}{Conversation with turn change}
     $\ldots d_{i-2}, y_{i-1}, d_{i-1}, y_{i} , d_i, y_{i+1}\ldots$
  \end{alertblock}
\end{frame}


\section{Summary Features}
\frame{\sectionpage}

%slide 3
\begin{frame}{Relative Turn Length}
\begin{enumerate}[<+->]\itemsep9pt
  \item Measure how long (in percent) the current turn length relative to his/her average turn length.
  \item {Computed as follow:
        \begin{itemize}
            \item $S_i$ to be the set of complete turns of speaker $s_i$ that are prior to the turn that $d_i$ is in.
            \item compute the average turn length up to the last turn
                \begin{block}{Average turn length }
                   $avg\_t_i = \frac{\sum_{t \in S_i} length(t)}{|S_i|}$
                \end{block}
            \item Compute relative turn length at the end of each dialog act
                \begin{block}{Relative Turn Length}
                     $rt_i =  \frac{length(t_i)} {avg\_t_i}$
                \end{block}
         \end{itemize}
        }
\end{enumerate}
\end{frame}


\begin{frame} {Relative Floor Control}
\begin{enumerate}
    \item Measure how long (in percent) the current speaker held the conversation floor. How dominate he/she is.
    \item Computed as follow:
         \begin{itemize}
              \item compute the total conversation time.
                 \begin{block} {Total Conversation Time}
                       $c_i = \sum_{t \in S_i \cup L_i} length(t)$
                 \end{block}
             \item  compute the total time of the current speaker relative to the total conversation time
                 \begin{block}{Relative Floor Control}
                     $\frac{\sum_{t \in S_i} length(t)} {c_i}$
                 \end{block}
         \end{itemize}
 \end{enumerate}
\end{frame}

\begin{frame} {Relative Floor Control}
%\begin{figure}[ht!]
%\centering
% \includegraphics[width=0.5\textwidth,width=8cm,height=8cm,keepaspectratio]{relative_control.png}
% \caption{Relative floor control\label{overflow}}
% \end{figure}
\end{frame}


\section{Evaluation}
\frame{\sectionpage}

% slide 5

\begin{frame}{Preprocessing}
  \begin{itemize}
    \item Removed 11 dialogue acts that were coded as other in switchboard.
    \item Skip the first 120 seconds of the conversation.
      \begin{itemize}
      \item Gives time for conversant to form the conversional image.
      \item Reduces the dialogue acts from 50633 to 37508.
      \end{itemize}
    \item Reduce data sparsity by collapsing 65 dialog acts into 9.
  \end{itemize}

  \begin{table}
     \begin{center}
     \begin{tabular}{l | l}
    \hline
Switchboard dialog acts &  Dialog act classes  \\
    \hline
sd,h,bf      & statement   \\
sv,ad,sv@    & statement - opinion  \\
aa,aa\^r     & agree accept \\
\%.\%-,\%@   & abandon      \\
b,bh         & backchannel  \\
qy,qo,qh     & question     \\
no,ny,ng,arp & answer       \\
+            & +            \\
o@,+@        & NA           \\
  \hline
\end{tabular}
\end{center}\vspace{-0.5em}
\caption{Mapping from dialog act to dialog act class}
\label{tab:mapping}
\end{table}
\end{frame}


\begin{frame}{ML Classifiers}
  \begin{itemize}
  \item Used random forests (N=200) to train and test the following models
  \begin{itemize}
    \item baseline 1: current dialog act label.
    \item baseline 2: current and previous dialog acts.
    \item summary model: just the summary features.
    \item full model: summary features and the current and previous dialog acts.
  \end{itemize}
  \item Evaluation was done using 10 fold cross validation.
  \item Run grid search to find the optimal hyper parameters.
  \end{itemize}
\end{frame}

% slide 4
\section{Results and Discussion}
\frame{\sectionpage}

\begin{frame}{Accuracy}
\begin{table}
\begin{center}
 \begin{tabular}{| l | l | l | c |}
    \hline
    Model & Accuracy & AUC & hyper parameters\\
    \hline
    Baseline 1      & 60.26\% & 0.63 & \scriptsize{max\_features=sqrt, max\_depth=7} \\
    Baseline 2     & 74.43\% & 0.79 & \scriptsize{max\_features=log2, max\_depth=9}\\
    Summary        & 66.14\% & 0.65 & \scriptsize{max\_features=sqrt, max\_depth=5} \\
    Full           & 76.05\% & 0.82 &\scriptsize{ max\_features=10, max\_depth=9}\\
  \hline
\end{tabular}
\end{center}
\caption{Accuracy, Area under the curve }
\end{table}
\begin{itemize}
\item Using only summary features is more accurate than Baseline 1.
\item The Full model is more accurate than Baseline 2.
\end{itemize}

\end{frame}

\section{Summary}
\frame{\sectionpage}

%slide 6

\begin{frame}{Conclusion and Future Work}
  \begin{itemize}
     \item Conclusion
       \begin{itemize}
          \item The experiment proved that summary feature improve turn transition predication
       \end{itemize}
  \item Future Work
     \begin{itemize}
        \item Combine summary feature with other local features:syntax,prosody.
        \item Test Simple moving average windows (5,10,20 turns)
        \item Test Exponential moving average.
        \item Convert other local features to summary feature.
     \end{itemize}
  \end{itemize}
\end{frame}



\begin{frame}{Perspective}{Disciplines for investigating energy topics}
    \begin{center}
    \begin{tikzpicture}[
      node distance=4.5em and .75cm,font=\small
    ]
    % flowboxes
    \node[flowbox] (physik) {
        \fbtitle{Model I}\vphantom{yÖ}
    \nodepart{two}
        \begin{minipage}{.16\textwidth}
            Current dial act            
        \end{minipage}
    };

    \node[flowbox,right=of physik] (technik) {
        \fbtitle{Model II}\vphantom{yÖ}
    \nodepart{two}
        \begin{minipage}{.16\textwidth}
            Current and prev dialog act
        \end{minipage}
    };

    \node[flowbox,right=of technik] (econ) {
        \fbtitle{Summary Model}\vphantom{yÖ}
    \nodepart{two}
        \begin{minipage}{.16\textwidth}
            Summary features 
        \end{minipage}
    };

    \node[flowbox,right=of econ] (society) {
        \fbtitle{All Model}\vphantom{yÖ}
    \nodepart{two}
        \begin{minipage}{.16\textwidth}
            All features 
        \end{minipage}
    };

    \uncover<2->{%
        \draw [decorate,decoration={brace,amplitude=10pt,mirror},ultra thick,jdblue]
            ($(technik.south west) + (-.2em,-1em)$) --
            ($(econ.south east)    + (+.2em,-1em)$) coordinate[midway,yshift=-3.8em] (midpoint-below);

        \node[flowbox] at (midpoint-below) (tech-econ)  {
            \fbtitle{Techno-economic modelling}\vphantom{yÖ}
        \nodepart{two}
            \begin{minipage}{.4\textwidth}
                \centering
                How much energy?
                For how much?\vphantom{yÖ}
            \end{minipage}
        };
    }
    \end{tikzpicture}
    \end{center}
\end{frame}
