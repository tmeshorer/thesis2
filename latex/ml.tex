\subsection{Classification Models}

    To test the contribution of the summary features, we used a binary classifier with
    $y_i$ as the outcome variable. We trained four models, which used the following sets of features:

    \begin{description}
        \item[baseline 1:] Predict turn transition based only on the current dialog act label.
        \item[baseline 2:] Predict turn transition based on the labels of the current and previous dialog acts.
        \item[summary model:] Predict turn transition using just the summary features.
        \item[full model:] Predict turn transition using the summary features and the current and previous dialog acts.
    \end{description}


We used random forests to build the binary classifiers $(N=200)$ \cite{Breiman01randomforests}. Random forests build an ensemble of decision trees during training, and during testing, each decision tree votes on the outcome.  Like decision trees, they can account for interactions between variables, such as making greater use of the summary features when the current speech act is not a question.  Random forests though are not as sensitive to overfitting and data fragmentation.

To find the optimal hyper parameters, we ran a grid search over the \textit{max\_features} and \textit{max\_depth} hyper parameters for each model. The hyper parameters search was done over $\{sqrt, log2, 10\}$ for \textit{max\_features} and $\{5, 7, 9\}$ for \textit{max\_depth}. When training the model, we used the optimal hyper parameters for each feature set.

We performed 10 fold-labeled cross validations.  We made sure that each conversation was entirely in a single fold. This way, each dialogue was entirely used for training or testing, but never for both at the same time.

\subsection{Metrics}
To evaluate our hypothesis, we use the trained model to perform prediction of the test data. We than compare the results against the truth values. The results are recorded in a confusion matrix.
Each row in the matrix represent the actual class and each column represent the predicted class.
The cell in the matrix are compute as follow :

\begin{tabular}{c >{\bfseries}r @{\hspace{0.7em}}c @{\hspace{0.4em}}c @{\hspace{0.7em}}l}
  \multirow{10}{*}{\rotatebox{90}{\parbox{1.1cm}{\bfseries\centering actual\\ value}}} &
    & \multicolumn{2}{c}{\bfseries Prediction outcome} & \\
  & & \bfseries p & \bfseries n & \bfseries total \\
  & p$'$ & \MyBox{True}{Positive} & \MyBox{False}{Negative} & P$'$ \\[2.4em]
  & n$'$ & \MyBox{False}{Positive} & \MyBox{True}{Negative} & N$'$ \\
  & total & P & N &
\end{tabular}

\begin{enumerate}
  \item True Positive - Labels which were predicted as positive and are positive
  \item False Positive- Labels which were predicted as positive but are in fact negative
  \item True Negative - Labels which were predicted as negative and are negative
  \item False Negative - Labels which were predicted as negative but are in fact positive
\end{enumerate}

Based on the confusion matrix, we compute the following metrics for each model:

\begin{equation}
Accuracy = \frac{TP + TN}{TP + TN + FP + FN}
\end{equation}
The accuracy measure how many instances were classified correctly out of the total instances.

\begin{equation}
Precision = \frac{TP }{TP + FN}
\end{equation}
The precision metric measure how accurate is the classifier for the positive instances. I.e. how many instances that were classified as positive, were in fact positive.

\begin{equation}
Recall = \frac{TP }{TP + FN}
\end{equation}
The recall metric measure how many positive instances were detected by the classifier.

\begin{equation}
F1 = \frac{2 \cdot Precision\cdot Recall}{Precision+ Recall}
\end{equation}
In order to have one measurement which encompass both recall and precision, we compute F1 which is the
harmonic mean of recall and precision.

Note that there is a trade off between recall and precision. I.e. if we want to increase recall we will reduce precision. To measure this tradeoff, we will use ROC (Receiver Operating Characteristic) curve.
The ROC curve measure the true positive rate (TPR) or recall, against the false positive rate (FPR). 

