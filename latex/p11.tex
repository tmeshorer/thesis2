We first analyzed the results in terms of accuracy: how often the models correctly predicted whether a turn transition occurred; in other words, how often the model predicts the correct value of $y_{i+1}$.
%
Table \ref{table:result} shows the results of training a random forest of 200 trees for each model using 10 folds cross validation. We see that using the summary features provides better accuracy than baseline 1, which use only the current dialog act ($65.54\%$ vs $62.79\%$). In addition, using the full model yields an improvement of over $1.08\%$ in the accuracy. In addition, baseline 1 has high precision, it has very low recall. Using only the summary model improves recall and decreases precision by less, leading to a higher F1 score and overall better performance. Using the full model improves precision, which means that dialog acts that were considered to lead to turn transitions are classified correctly. If we use the full model, we lose precision (over baseline 2 model), but gain recall,
leading to the highest F1 score and the best performance.
%
\begin{table}[ht!]
\begin{center}
\begin{tabular}{lrrrrr}
\toprule
{} &  Accuracy &        F1 &  Precision &    Recall &   AUC \\
\midrule
baseline 1 &  0.627938 &  0.578149 &   0.749824 &  0.470491 &  0.659992 \\
baseline 2 &  0.748900 &  0.748792 &   0.818479 &  0.690012 &  0.811185 \\
summary    &  0.655471 &  0.693203 &   0.672255 &  0.713612 &  0.694657 \\
full       &  0.757515 &  0.775993 &   0.775099 &  0.778319 &  0.837863 \\
\bottomrule
\end{tabular}
\end{center}
\caption{Precision, recall and F1 results using Random Forests }
\label{table:result}
\end{table}


The effect can also be seen in figure \ref{auc}, which shows the ROC curves and the AUC for each
model. We notice that the AUC of the summary model is better than baseline 1 model ($0.69$ vs $0.66$), and when adding the summary features to the local features (the full model), we see the AUC improves ($0.84$ vs $0.81$). This suggests that while the discrimination facility of the summary features is lacking (AUC $<0.7$), adding them to a classifier that uses local features (full model) yields better results than the baseline.
%
 \begin{figure}[ht!]
 \centering
 \includegraphics[width=\textwidth]{../scikitlearn/figures/roc.pdf}\vspace{-1.5em}
 \caption{ROC curves and AUC of the different models \label{overflow}}
\label{auc}
 \end{figure}


\begin{table}[ht!]
\begin{center}
\begin{tabular}{lrrrrr}
\toprule
{} &  accuracy &        f1 &  precision &    recall &   roc\_auc \\
\midrule
all        &  0.765718 &  0.787481 &   0.774492 &  0.801152 &  0.848495 \\
baseline 1 &  0.627938 &  0.578149 &   0.749824 &  0.470491 &  0.659992 \\
baseline 2 &  0.748834 &  0.748210 &   0.819280 &  0.688687 &  0.811097 \\
summary    &  0.679110 &  0.713018 &   0.692053 &  0.735588 &  0.726475 \\
\bottomrule
\end{tabular}
\end{center}
\caption{Precision, recall and F1 results using Random Forests }
\label{table:result}
\end{table}


\subsection{Sensitivity to measurement start time}

In Meshorer and Heeman \cite{Meshorer2016UsingPS} we assumed that it take 120 seconds for the conversational image to form. To validate this assumption here, we computed the AUC for the different model using different starting time. Table \ref{table:starttime} shows that the AUC values stay consistent with using different start times. This shows that the summary features predictive strengh is not affected by the start time.

%
\begin{table}[ht!]
\begin{center}
\begin{tabular}{lrrrrrrr}
\toprule
{} & 0 & 15 & 30 & 45 & 60 & 120 & 180  \\
\midrule
baseline 1 & 0.659992 & 0.661070 & 0.661265 & 0.660901  & 0.660261 & 0.659855 & 0.660598  \\
baseline 2 & 0.811185 & 0.812167 & 0.812493 & 0.812012  & 0.811513 & 0.809206 & 0.806845  \\
summary    & 0.694657 & 0.695172 & 0.694351 & 0.694907  & 0.695737 & 0.691010 & 0.692128  \\
all        & 0.837863 & 0.838777 & 0.838533 & 0.838016  & 0.836148 & 0.831940 & 0.828085  \\

\bottomrule
\end{tabular}
\end{center}
\caption{ AUC Score in relation to the start of the dialog }
\label{table:starttime}
\end{table}



