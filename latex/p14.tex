In this work, the local features that we considered in our baseline model were just the last two speech acts.  Other work on turn-taking prediction use a richer set of local features, such as
%
syntactic \cite{ duncan1972some, sacks1974simplest, ford1996interactional, de2006projecting, magyari2012prediction, atterer2008towards}, prosodic \cite{ duncan1972some, ford1996interactional, shriberg2000prosody, ferrer2003prosody, de2006projecting, reed2009units, raux2012optimizing, hariharan2001robust, atterer2008towards}, pragmatic \cite{ ford1996interactional, garrod2015use, raux2012optimizing}, semantic \cite{raux2012optimizing} and non-verbal \cite{kendon1967some}. In future work, it would be good to evaluate the contribution of our summary features with a richer set of local features.

In our work, we evaluated our model on the Switchboard corpus.  In future work, it would also be good to evaluate our summary features on other corpora, especially task-based dialogues.  Tasks in which there is a difference in the role of the user and speaker, such as in Trains \cite{HeemanAllen95:cdrom}, should benefit from modeling the past turn-taking behavior of each speaker.

In our work, we computed the relative turn length and relative speaker control using the turn length avrage as computed over all the previous turns. In future work , it would be interesting to use simple moving average (measured over multiple window width) as well as exponential moving average. 

More generally, the summary features introduced in this work represent just one aspect of the conversational image of the user. Future work should try to ``summarize'' other local features by creating
the average value of a local feature over past turns. For example, we can compute relative speech rate, or relative use of stereotyped expressions.
