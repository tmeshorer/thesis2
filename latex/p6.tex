This section present work related to turn transition prediction in both human-human conversation as well human-computer conversations. 
The importance of turn transition was first established in the seminal work of \cite{sacks1974simplest} which form the foundation to the field
of conversation analysis. In addition, other early theories recognized the importance of signalling (using both prosodic, syntactic and even non
verbal ques) the upcoming release of the conversation floor by the speaker.  Since most of the studies in human-human conversations study the affect of
features derived from last few utterance on projection of turn transition, the research presented in this disassertion complement both the turn
allocation system as presented by \cite{sacks1974simplest} and the signaling theory as presented in \cite{duduncan1972some}. 

First generation SDS systems used a timeout based approach to turn taking. This simple approach causes system barge ins (in which the system wrongly
identify intra turn pauses as turn transition) as well as large gaps between turns (in which the user finish it turn, but wait until the timeout trigger).
Second generation SDS system incorporating more advanced machine learning models which uses various local features - prosodic, syntactic , pragmatics to 
predict the release of conversation floor. As before, our work complement the local features used by the current machine learning models and suggest that
those model can be improved if augmented by summary features.        
   
