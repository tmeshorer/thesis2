Duncan \cite{duncan1972some} argued that speakers signal when they want the listener to take the turn and presented six signals used by the speaker to accomplish this: intonation, drawl on the final syllable, body motion, sociocentric sequence, drop in pitch or loudness, and syntax. Kendon \cite{kendon1967some} included a gaze as an additional signal for turn transition. As the signals are local in nature (derived from the last turn), our summary features complement them.


Turn allocation was introduced in the seminal work by Sacks, Schegloff, and Jefferson \cite{sacks1974simplest}, who observed that conversations are ``one speaker at a time'' and gaps between turns as well as speaker overlaps are kept to a minimum. To satisfy these constraints, Sacks et al.~suggested an ordered set of rules for turn allocation:
%
(a) current speaker selects the next conversant; (b) if the current speaker did not select, any of the listeners can self select; or (c) if neither of the previous two cases apply, the current speaker continues.
%
For the first rule, Sacks et al.~suggested that the current speaker uses adjacency pairs as the main apparatus for selecting the next speaker. Hence, we recognized the importance of dialog acts in turn allocation and chose them as the atomic turn components.
% Peter says we should really have a better justification of why we use acts as our building block
In addition, our work might impact the second rule, in which the conversant self selects. While Sacks et al.~suggested that the first starter is the next speaker, we suggest that a conversant might use the conversational image of the speaker and of themselves when self selecting. For example, a controlling speaker (with a high relative floor control score) has a better chance to gain control of the conversation floor when self selecting.

In addition to the turn allocation system, Sacks et al.~also suggested that turn construction units (TCU) should support projection of turn ends by the participants. The projectability attribute was later extended to other features of the speaker's utterance: (syntactic \cite{sacks1974simplest}, prosodic \cite{ford1996interactional} and pragmatic \cite{ford1996interactional,ford2001intersection}). Our work augments the local utterance features with summary features that can be used to improve projectablity.

Selfridge and Heeman \cite{SelfridgeHeeman10:acl} work on turn bidding suggests that each conversant measures the importance of their utterance when negotiating the right to the conversation floor. They work
is based on the fact that 1) The importance of speaking is the primary motivation behind turn-taking behavior, and (2) conversants use turn cue strength to bid for the turn based on this importance. Out work suggest that contestants might use summary feature to cue the strength of the bid. For example conversant will take turn longer than their average if they want to signal turn importance.



Wilson et al \cite{wilson2005oscillator} explored the role of entrainment in turn taking. They proposed there is entrainment of endogenous oscillators in the brains of the speaker and the listener on the basis of the speaker syllabus production. In their model, the speaker and the listener are counter-phased so that speech overlaps and gaps are minimized. Although our work does not imply cyclic synchronization between speaker and listener, we do suggest that each conversant creates a conversation image of the other conversant and uses it during turn transition.

The importance of using dialog acts was emphasized by a very recent study of Garrod and Pickering \cite{garrod2015use}. The study suggested that turn production is a multi-stage process in which the listener performs simultaneous comprehension of the existing turn as well as production of the new turn content. They suggested that the first step in the process is dialog act recognition, which is done as soon as possible and acts as the basis for the listener's turn articulation and production. In our study we use dialog act as the main turn component.
