To evaluate the importance of the summary features in predicting turn transitions we used the 2010 version of the switchboard corpus \cite{calhoun2010nxt} which is based on the original release \cite{225858} of the switchboard corpus.

The switchboard corpus is the first large collection of phone conversations and was collected in 1990-1991. The original corpus was composed of 2483 calls involving 520 speakers. Each call ranged from 1.5 minutes to 10 minutes, with an average length of 6 minutes. Conversations involved a randomly chosen topic between two randomly selected speakers. The corpus was later improved and was released as part of the Penn Treebank 3 corpus, which included 650 conversations. The current release used for this research includes 642 conversations and just over 830000 words.

The current corpus contains different annotations over the original data. We used the turns annotation to assign turns to speakers and mark the turn start and end time points for each speaker. We used the dialog act annotation to assign dialog acts to turn such that a turn contain all the dialog acts which are contained within the turn time slot. Next we use the token annotation to assigned tokens to dialog acts and to compute the overall number of words within each dialog act as well as the dialog act start and end time.
