To evaluate the importance of the summary features in predicting turn transitions we used the 2010 version of the switchboard corpus \cite{calhoun2010nxt} which is based on the original release \cite{225858}.

The switchboard corpus is the first large collection of phone conversations and was collected in 1990-1991. The initial goal of the corpus is to facilitate speech research by providing pre recorded audio files of day to day conversations. The original corpus was composed of 2483 phone calls involving 520 speakers. Each call ranged from 1.5 minutes to 10 minutes, with an average length of 6 minutes. Conversations involved a randomly chosen topic between two randomly selected speakers. The corpus was later improved and was released as part of the Penn Treebank 3 corpus, which included 650 annotated conversations. The current release used for this research includes 642 conversations and just over 830000 words.

The current corpus contains multiple types for annotations for each conversation. We used the turns annotation to assign turns to speakers and mark the turn start and end points. We used the dialog act annotation to assign dialog acts to turns such that each turn interval contain all the dialog acts that occured between turn start and turn end. Next, we use the token annotation to assigned words to dialog acts and to compute the overall number of words within each dialog act.
