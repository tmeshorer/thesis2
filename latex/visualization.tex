In this section we will use visualization method in order to explore the data variables
distribution and relationships.

\subsection{Variables}

After pre processing the data, the data set contains the following variables

\begin{table}[ht!]
\begin{center}
\begin{tabular}{llllrr}
\toprule
Variable &  Description & Type &\\
\midrule
     Previous Dialog Act & the dialog act before the current one  & categorical\\
     Dialog Act & the current dialog act & categorical \\
     Length & length of the current dialog act in seconds & seconds \\
     Relative Turn Length (RTL)  & Relative turn length as defined in \ref{sfeatures} & percent \\
     Relative Time Control (RTC) & Relative time control as defined in \ref{sfeatures} & percent \\
     Turn Change & 1 if there was a turn change after this dialog act & binary \\
\bottomrule
\end{tabular}
\end{center}
\caption{Data Fields}
\end{table}


\subsection{Dialog Acts}

Figure \ref{dactsizefig} shows the percent for each dialog act. We can observe that the majority of dialog acts are statements, backchannels and opinions. This is true to the nature of the switchboard corpus
which consists mainly of casual conversations.

 \begin{figure}[ht!]
 \centering
 \includegraphics[width=0.5\textwidth]{../scikitlearn/figures/precent_dacts.png}
 \caption{Dialog act relative count\label{overflow}}
 \label{dactsizefig}
 \end{figure}

Figure \ref{dactssecsbyname} shows an histogram of the length in seconds for each dialog act. It is evident that the distribution is left tilted.

 \begin{figure}[ht!]
 \centering
 \includegraphics[width=\textwidth]{../scikitlearn/figures/grid_secs_by_da_name.pdf}
 \caption{Distribution of dialog act length in seconds \label{overflow}}
 \label{dactssecsbyname}
  \end{figure}

 Figure \ref{l3} shows the and histogram for each dialog act length conditioned on the occuruce of a turn change. We can observe that the length distribution is the same, however when a turn change occur, some
 dialog acts tends to be longer than regular.

\begin{figure}[ht!]
 \centering
 \includegraphics[width=\textwidth]{../scikitlearn/figures/grid_secs_by_da_name_by_tchange.pdf}
 \caption{Dialog act length in words conditioned on turn change\label{overflow}}
 \label{l3}
 \end{figure}

In figure \ref{l4} , we measure the probability that a dialog act will lead to turn change. We can observe that the majority of back channels and question will lead to a turn change.

\begin{figure}[ht!]
\centering
\includegraphics[width=0.5\textwidth]{../scikitlearn/figures/barplot_da_prob_to_tchange.png}
\caption{Dialog act probability of turn change\label{overflow}}
\label{l4}
\end{figure}


\subsection{Relative Turn Length}

Figure \ref{l5} shows the distribution of the relative turn length for each dialog act. We can observe
that

\begin{figure}[ht!]
\centering
\includegraphics[width=\textwidth]{../scikitlearn/figures/grid_precent_secs_sofar_by_da_name.pdf}
\caption{Relative turn length for each dialog act\label{overflow}}
\label{l5}
\end{figure}


\subsection{Relative Floor Control}

Figure \ref{l6} shows distribution of floor control or each dialog act conditioned on a turn change. We can observe that regardless of the dialog act, most distributions are normal with mean of 50\%

\begin{figure}[ht!]
\centering
\includegraphics[width=\textwidth]{../scikitlearn/figures/grid_timecontrol_by_da_name_by_tchange.pdf}
\caption{Relative flow control conditioned on turn change\label{overflow}}
\label{l6}
\end{figure}


